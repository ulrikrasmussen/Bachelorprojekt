% vim:spell:spelllang=en_gb:
\section{Disposition}


\section{Join-calculus}

Join-calculus is a process calculus for modelling distributed systems,
communicating in an asynchronous way. It is described in
\cite{fournet1996reflexive} using an extension of the \emph{Chemical Abstract
Machine}\cite{berry1989chemical} (or CHAM) as a model for computation. This
extended model is called the \emph{reflexive} CHAM.

The join-calculus can be seen as a concurrent extension of functional
programming, and the calculus is closer to an actual programming language than
other process calculi, like the $\pi$-calculus. The only values are
\emph{names}, which act both as port (or channel) names, and as transmitted
values. For instance, if $x$ and $y$ are names, $\S x<y>$ means that we send
$y$ over $x$. A message $\S x<y>$ is also called an \emph{atom}, and multiple
atoms can be combined into \emph{molecules} using `$|$'.

In this section, we will briefly explain the workings of the reflexive CHAM
using examples.  We denote a given state of the CHAM by $\mathcal{R} \vdash
\mathcal{M}$, where $\mathcal{M}$ is a multiset of molecules, and $\mathcal{R}$
is a set of molecule reaction rules. The CHAM states are related by a
bidirectional \emph{heating relation}, `$\HeatCool$'. Two states related by
`$\HeatCool$' are structurally equivalent, and therefore interchangeable.  One
equivalence states that we can combine the molecules of a solution in all
possible ways:

$$
  \vdash \S x<> | \S y<> \ \HeatCool{} \vdash \S x<>, \S y<>
$$

Appropriate application of heating and cooling allows molecules to take part in
reactions, that subsequently consumes them from the solution.

If there exists a rule in $\mathcal{R}$ that matches a given molecule, the rule
can be applied:
$$
  \S x<a> | \S y<> \triangleright \S z<a> \vdash \S x<v> | \S y<>
  \React \S x<a> | \S y<> \triangleright \S z<a> \vdash \S z<v>
$$
Note that reductions are defined up to substitution of the transmitted values.

It is possible for a molecule to define new reduction rules. The syntax of
molecules is extended to include definitions:
$$
  \jdef{\S x<> \triangleright \S y<>}{\S x<> | \S z<>}
$$

A heating relation adds the definition to the set of rules. If the above
definition is placed in $\mathcal{M}$, we get
$$
 \emptyset \vdash \Def \S x<> \triangleright \S y<> in \S x<> | \S z<>.
 \HeatCool \S x<> \triangleright \S y<> \vdash \S x<> | \S z<>
 \React \S x<> \triangleright \S y<> \vdash \S y<> | \S z<>
$$
