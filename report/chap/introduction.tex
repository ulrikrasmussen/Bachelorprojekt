% vim:spell:spelllang=en_gb:
\section{Problem statement}

\section{Join-calculus}

\fixme{we may need a clearer definition of ``mobile'' and ``distributed''}
Join-calculus is a process calculus for modelling mobile and distributed
systems, communicating in an asynchronous way. It is described in
\cite{fournet1996reflexive} using an extension of the \emph{Chemical Abstract
Machine}\cite{berry1989chemical} (or CHAM) as a model for computation. This
extended model is called the \emph{reflexive} CHAM.

The join-calculus can be seen as a concurrent extension of functional
programming, and the calculus is closer to an actual programming language than
other calculi, like the $\pi$-calculus. The only values are \emph{names}, which
act both as port (or channel) names, and as transmitted values. For instance, if
$x$ and $y$ are names, $\S{x}{y}$ means that we send $y$ over $x$. A message
$\S{x}{y}$ is also called an \emph{atom}, and multiple atoms can be combined
into \emph{molecules} using `$|$'.

We denote a given state of the CHAM by $\mathcal{R} \vdash \mathcal{M}$, where
$\mathcal{M}$ is a multiset of molecules, and $\mathcal{R}$ is a set of
molecule reaction rules. The CHAM states are related by a bidirectional
\emph{heating relation}, `$\rightleftharpoons$'. Two states related by
`$\rightleftharpoons$' are structurally equivalent, and therefore interchangeable.
One equivalence states that we can combine the molecules of a solution in all
possible ways:
$$
  \vdash P_1 | P_2 \ \rightleftharpoons{} \vdash P_1, P_2
$$
where $P_1, P_2$ are molecules. Appropriate application of heating and cooling
allows molecules to take part in reactions, that subsequently consumes them from
the solution.

If there exists a rule in $\mathcal{R}$ that defines a reduction for a given
molecule, the rule can be applied:
$$
  \S{x}{a} | \S{y}{} \triangleright \S{z}{a} \vdash \S{x}{v} | \S{y}{}
  \longrightarrow \S{x}{a} | \S{y}{} \triangleright \S{z}{a} \vdash \S{z}{v}
$$
Note that reductions are defined up to substitution of the transmitted values.