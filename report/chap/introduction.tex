% vim:spell:spelllang=en_gb:
% vim:set textwidth=80 fo+=tcroql:

We have chosen to examine the usefulness of using the join calculus for
programming embedded, distributed devices, such as the sensors in a sensor
network, or even the robots in the emerging fields of self reconfiguring
robotics and swarm robotics in the long run.

The join calculus is a process calculus that seeks to bridge the gap between the
$\pi$-calculus and the actual programming languages that contain elements of
distributed execution. The defining article, \cite{fournet1996reflexive}
especially refers to the languages Actor and Obliq.

The intended goal of this calculus is to define a way of expression that is more
formally defined than the existing programming languages and that does not rely
on synchronous rendez-vous communication, as this is often difficult to
implement in a distributed setting.

Because of this, message exchange in the join calculus is asynchronous and
provides means for both communication and synchronisation.

The measurement of a programs \emph{feasibility} for a particular purpose is
definitely a debatable subject. In this report however we have chosen to
investigate by constructing sample programs from the problem domain and discuss
the patterns used and how well the \fixme{Wierd sentence x\_x}{language lends
itself to practical implementation}.

In order to evaluate to be able to evaluate the programs we create we have put
together our own tailored programming language and implemented an interpreter
capable of executing it. The programming language is based on the join calculus,
and apart from including the core of the formalism we have also deemed some
proposed extensions relevant for our problem, as they introduce both the notion
of mobility, pattern matching over algebraic datatypes, and a notion of time.
In order to gracefully include concepts from these different sources we have had
to alter their semantics to some degree to make them coexist without interfering
with each other.

The formal operational semantics \fixme{isn't it a bit bold to say?}{that the
interpreter implements} is also documented in this report.

Finally we discuss \fixme{do we really?}{the process of constructing the
programs and} the benefits and detriments that we have identified of the
programming language. We will only examine the join calculus in an introspective
way. Therefore we will only present join solutions to the problems we pose. One
could imagine it being useful to have dual implementations in other languages
for comparison, but that is beyond the scope of this report, as it implies that
we need to make certain choices of programming languages and concurrency
libraries, and make a very thorough attempt at constructing programs that are
representative of the expressiveness of the given languages.
