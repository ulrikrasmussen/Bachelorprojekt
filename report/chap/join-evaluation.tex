% vim:spell:spelllang=en_gb:
% vim:set textwidth=80:

In order to assess to some degree the usefulness of programming distributed
devices in the style of the join calculus, we have decided to investigate by
constructing miniature examples, which we will then evaluate upon.

As the scenarios we intend to model involve communication mostly over
unreliable channels, it is interesting to look into the ways of
expressing communication protocols at the level of the application,
that addresses the issue of guaranteed delivery.

Therefore we have programmed a use case featuring a sensor network,
and a general purpose rudimentary library for guaranteed message
delivery to deal with the simulated packet losses between the sensors.

Lastly we include a discussion of some of the implications of the distributed
model of computation and identify potentially interesting subjects for future
work.


\begin{comment}
Disposition:
------------
*Present usecase, its purpose, inspiration.
*Introduce scenario.
  - Assumptions of underlying system.
  - number of msc etc.
*Show overview pseudo-code and describe implementation.
*Discuss generalisability of the program.
  - Argue that more general server-sensor queries may be constructed,
  - And that more general ad-hoc registration is possible.
\end{comment}

In this chapter we present our implementation of a simplified version of a
distributed query system applied to a sensor net. Essentially it is just an
instance of the \fixme{insert citation!.. think its in join-tutorial.}{applet
server} pattern, although a bit more involved.

The example is heavily inspired by \cite{bonnet2001towards}, and
evolves around a factory warehouse setting where each item has a
stick-on sensor that measures the temperature. There are also
sensors on the walls and ceilings. All sensors have a unique id available to them.

Apart from sensors collecting nodes exist, also called \emph{servers}, that
distribute the queries to the sensors. The term \emph{client} refers to the machine
that distributes queries to the collecting nodes and receives the output of the queries.

\subsection*{Distributed queries in sensor networks}

In \cite{bonnet2001towards}, the concept of sensor databases is
introduced, in which a set of sensors is queried using an extension of
SQL, enabling the user to extract very specific datasets from the
network. The queries are executed in a distributed fashion, so the
sensors only transmit data relevant to the query.  For high-resolution
sensors generating vast amounts of data, processing the data locally
is cheaper than transmitting it for offline processing, leading to a
more efficient usage of resources for bandwidth and battery
constrained devices.

The COUGAR project\cite{COUGAR} has already done a lot of research in this
area, and a complete system for executing declarative queries in a distributed
fashion has already been created. However, it would be interesting to see if a
join-calculus based language would prove useful for expressing programs that
essentially do the same as a distributed query. \cite{bonnet2001towards}
mentions that one of the challenges of implementing distributed queries was
handling the asynchronicity of events, which we hope will be easier to deal
with in a join-calculus based language.

\subsection*{Example scenario}

Due to the lack of data structure libraries and the general ineffeciency of the
interpreter, we only consider a scenario with one collecting node and two
sensors, statically positioned. No new sensors will arrive, but sensor hardware
failure is simulated.

Since our example only features a single collecting node, the server and client
are merged into one machine for simplicity.

The query program that is uploaded to the sensors simply measures the local
temperature and reports the average every $n$'th time instant.  However, if the
sensor registers a temperature above a given threshold it will report it
immediately.

Every $m$'th time instant the server outputs the received temperature readings
along with an assessment of the age of each reading. If a reading isn't updated
within a given time interval, the program assumes that the sensor is broken and
prompts the operator to replace it.

\subsubsection*{Assumptions about the underlying system}

We will assume a network topology with a central processing node
which is connected to the sensors in a star pattern, using
unreliable communication channels (i.e. some sort of radio protocol).
We do not assume that the runtime system ensures delivery of messages,
but we do assume that any messages that \emph{do} get delivered are
ensured not to be corrupted in any way.

However, since we don't have a mature, compiled language, we will not focus on
memory and processor constraints, since the resource usage depends a lot on the
implementation. We may assume, however, that the actual usage of memory
resources is in some way proportional to the number of active messages and join
patterns on each device , and that the usage of computational resources is
proportional to the number of reduction steps required to arrive at a result.

It is also assumed that there is a built in name-exchange facility,
such that the network can be bootstrapped. In our scenario, this is a
global key-value store that can be accessed programmatically from any
device through the API messages \verb!register! and \verb!search!. A
call to one of these messages always succeeds. In a real-world
scenario, bootstrapping an ad-hoc network may be more involved, since
we cannot rely on a global, centralized non-failing database. Most likely one
could be moderatly succesfull by implementing this name-exchange api by
performing regular broadcasts at the \fxnote*{is it called this?}{link-level}
of the network to anyone willing to listen.

The only type of communication that is assumed to be reliable is location
migration, as there is no sensible way to reproduce a location, if it is lost
in transit.

\subsection*{Implementation}

Conceptually, a query program is structured in two parts: One that
treats the sensor data inside the sensor itself, and a second part
that runs on the server node and collects and optionally filters the data from
the sensors and eventually aggregate these to the client.

Sensor:
\begin{verbatim}
def
 ..library definitions..

 or mkSensor<mscId, handshake> |>
    def
        query[
            killQuery<> |> halt<>
         or readTempLoc() |> {return readTemp(mscId) to readTempLoc}
         in 0
        ]
     in handshake<mscId, readTempLoc, query>
 or connect<> |>
      { match search("server") with
          Nothing -> { run 1:connect<> }
        | Just handshake -> { let mscId = machineId()
                           ; run mkSensor<mscId, handshake> } }
 in connect<>
\end{verbatim}

\fixme{prepare for presentation..}

Server:
\begin{verbatim}
def
  [library definitions]
 or mkQuery(threshold, readTemp, callback) |>
                 { let num = 10
                 ; run def query[
                     collect<n, tAkk> |>
                       { let t = readTemp()
                       ; [if threshold <= t then
                           { run callback<Above(t)> & 1:collect<1,t>}
                          else if n=num then
                           { let avg= (t+tAkk)/(1+num)
                           ; run callback<Avg(avg)> & 1:collect<1,avg> }
                          else
                           { let tAkk' = t + tAkk
                                 nextT = n+1
                           ; run 1:collect<nextT, tAkk'>
                           }]
                       }
                  or migrate<locNm> |> { do go(locNm); let t = readTemp(); run collect<1,t>}
                  in 0 ] in {return migrate to mkQuery}
                 }

 or mkServer<> |>
  {
  ; let cl = mkClock(0)
  ; run def
            sensorHandshake<mscId, readT, qLoc> |> {
              match mscId with
                "Sensor_A" -> {let query = mkQuery(20, readT, sensorACallback)
                              ; run query<qLoc>}
              | "Sensor_B" -> {let query = mkQuery(22, readT, sensorBCallback)
                              ; run query<qLoc>}
            }
         or sensorACallback<temp> & collected<Sensors(aT, aN, bT, bN)> |>
                                   { let time = cl()
                                   ; run collected<Sensors(temp, time, bT, bN)>
                                   }
         or sensorBCallback<temp> & collected<Sensors(aT, aN, bT, bN)> |>
                                   { let time = cl()
                                   ; run collected<Sensors(aT, aN, temp, time)>
                                   }
         or printCollected<>
          & collected<data> |> { [output data pretty printed]} & 5:printCollected<>
                                                    & collected<data>
         in collected<Sensors(Above 0,0,Above 0,0)>
          & { do register("server", sensorHandshake) }
          & 5:printCollected<>
  }
 in mkServer<>
\end{verbatim}
\subsection*{Discussion}

\subsubsection*{Doing I/O}

\fixme*{refer to some introduction of the way we handle external input, then delete subsection}{%
We will begin by defining how the sensors will read data from the
outside world. Since the join-calculus has no way of doing I/O, we
have to define some ``magic'' messages that forms an API for reading
temperature values.
}
We read the temperature via explicit probing using the name \verb+read_temperature+.

\subsubsection*{A programmable sensor}

\fxnote{Preserved because it has insights.}
Note that in this use case, we exploit the ability to halt
locations to ensure that any processes that may have migrated to
the sensor is killed when we loose connection to the server. We
also "wrap" our exported API calls in messages defined in the
\verb!here! location. This ensures that the exposed interface is
invalidated whenever the location is killed, ensuring that no
process can continue reading from the sensor from an external
location (which would destroy the purpose of migrating queries to
the sensors in the first place).

\subsubsection*{Programming the central nodes}

\emph{stuff for generalising}
Initially, query programs enter the server node from a client node
through a set of names, that the server node exports.
Then, as sensors register themselves with the server node, the query
programs are uploaded from the server to the sensors.

In order for this to work, a correct query program is supposed to
behave in a certain way with regards to distribution and internal
structure, because process migration in the join calculus only happens on
the initiative of the process that is to migrate, and cannot be controlled
or observed by the surrounding environment.
----
When a new sensor appears in the network, it enters the registration
phase, where the sensor exports the names used for accessing the
sensor hardware, and the filter-part of the query program migrates to
the sensor.
While the sensor is active and registered, any exchange of sensor data
happens only on the initiative of the filter part of the query
program.

Since the nature of queries vary it is up to the query program to determine
what to do if the connection to the server is lost.


% vim:spell:spelllang=en_gb:
% vim:set textwidth=80:

\section{A reliable communication protocol}

In order to deal with the imperfections of basic radio and network
communications we have implemented a rudimentary library that enables reliable,
ordered delivery between two communicating parties.

The communication protocol works by the same basic principles as does TCP. In
particular it gives the same guarantees about the message exchanges, namely
those of

\begin{itemize}

\item Retransmission of lost messages

\item Ordered message treatment % currently implemented by assuring that no
%message can be sent before the

\item Duplicate messages are discarded

\end{itemize}

Central to the workings of the protocol is the bookkeeping that takes care of
ordering the messages, by labeling each message to be sent with a sequence
number. Only when the recipient acknowledges the reception of a message by
sending an acknowledgement message that carries the same sequence number as the
message of the sender did, the transmission of new messages can continue.  An
overview of the protocol is given below.

\paragraph{The sending side:}
\begin{enumerate}

\item \label{retrans:SYN}
The server sends a \texttt{Syn}-message to initialise the connection

\item If a \texttt{SynAck} message is received within a given time window, the
connection is established and the channel is marked clear, otherwise repeat \ref{retrans:SYN}

\item \label{retrans:Send} When the sender wants to communicate, it increments
the sequence number and sends the message along with the sequence number. Then it
removes the clear mark of the channel, effectively blocking the channel.

\item If a \texttt{DataAck} message is received within the given time window,
and its sequence number matches the sender's, the channel is marked clear.
Otherwise the received \texttt{ACK} is discarded, and no further state change
occurs, or if no \texttt{ACK} was received, the message gets retransmitted.

\end{enumerate} \paragraph{The receiving side:} \begin{enumerate}

\item Upon reception of a \texttt{Syn}-message, send a \texttt{SynAck} and set
the sequence number to zero.

\item Upon reception \texttt{Data}-message with higher sequence number,
increment the sequence number and send a \texttt{DataAck}-message containing the
result of computation triggered by the reception if any. Also store the result,
in case the \texttt{DataAck} needs to be retransmitted.

If the sequence number of the \texttt{Data}-message is equal to the currently
recorded sequence number, the last sent \texttt{DataAck} is retransmitted, and
no further computation is triggered.

\end{enumerate}

\paragraph{Shortcomings.} This sketch of a communication protocol is indeed very
basic, and we have made no attempt at refining it further.  This is because we
believe that it is obviously possible to accomplish within the expressive power
of the programming language, looking at TCP for inspiration on optimisations,
and adding more integrity checks to ensure that one buggy participant doesn't
ruin the other.

\subsection{Usage of the library}

Actually using the library as an underlying data communication layer is
accomplished by \emph{proxying} the names used for communication through the
protocol library.  To make this clearer, consider this usage example of a thin
client using server for computation:

\paragraph{Server program}

\begin{verbatim}
def
    include(retransmit.join)
    bigComputation(x) |> {.. return f(x) to bigComputation}
 in
    { let exposedCom = listen(Sync(bigComputation));
    ; register("bigComputation", exposedCom)
    }
\end{verbatim}

\paragraph{Client program}

\begin{verbatim}
def
    include(retransmit.join)
 in
    { let bigCom = search("bigComputation")
    ; match bigCom with
        Just bigCom' -> { let bigComProx = initTrans(Sync, bigCom')
                        ; do print(bigComProx(100))
                        }
        Nothing      -> { do print("Couldn't find bigComputation") }
    }
\end{verbatim}

Above the technique of proxying is applied to yield two new names,
that each represent an end of the communication channel.

The proxy at the server end is the name returned by the synchronous call to
\texttt{listen}. This is the name that is to be published to allow for
communication. Feeding this name into the \texttt{initTrans} call yields the
final name that the client application interacts with.

Notice that beyond the call to \texttt{initTrans} it is impossible to
distinguish a proxied name from an unproxied name, and using the library
involves no extra complexity, which furthers the case that usage of the join
calculus shouldn't necessarily be limited to modelling concurrent computation
inside a single physical unit, but is also a decent option for a distributed
system, even though the communication might be flawed.


% vim:spell:spelllang=en_gb:
% vim:set textwidth=80 fo+=tcroql:

\section{General discussion}

While we have been able to successfully produce sample programs without the need
to resolve to using cumbersome encodings, and demonstrated the rather
transparent way the proxying pattern could be applied to deal satisfactorily
with the issues of unreliable communication, some fundamental implementation
issues need to be solved before the use of the join calculus as a foundation for
distributed programming can be said to be realistic.

\subsection*{Migration problems}
When a location migrates, every location it has ever exported a name to knows
which physical machine to send new messages to.  Since the definition site of a
name has no given means to know the extent that a name escapes into the network
once it has escaped its scope, it is impossible to notify every site that is
knowledgeable of the name that its definition site has moved to a different
physical machine.

The following code exemplifies the problem:
\begin{verbatim}
def
    get_trig<k> & app_trig<t> |> k<t>
 or server[
       trigger<t> |> t<> & 10:trigger<t>
    in get_trig<trigger>
    ]
 or applet[
       do_work<> |> ...
    or trigger<> |> ...
    in { run app_trigger<trigger>
       ; do go(client_1)
       ; run do_work<>
       ; do go(client_2)
       ; ...
       ; do go(client_n)}
    ]
 or site_1[...] or ... or site_n[...]
\end{verbatim}

If we assume that the locations \verb+site_1...site_n+ represent different
physical machines, some special means is needed to route the trigger messages to
the applet as it migrates.

\vspace{1em}

If one were to speculate in possible solutions to this problem one could imagine
either a central name server or a distributed key-value store that maps names to
machines. This seems horribly ineffective, because it seems that a consequence
is that everybody needs to know where everybody else is, and either produce a
single point of failure or a lot of network traffic containing administrative
overhead whenever locations migrate.

\vspace{1em}

An entirely different approach would be to associate a name to the physical
machine that hosted the escaping context, and require the migrating locations to
implement a scheme manually that solves the problem satisfactorily in the given
scenario. This might possibly work if the actual communication requirements in
most applications are easier to handle than the general name routing problem.
However, further investigation of this is beyond the scope of this thesis, but
could be considered an interesting future work.

\subsection*{Synchronised time}

Another unrealistic aspect of the language is the model of time that it
exhibits. In the current model time progresses only when every machine has
finished its time instant, such that it is in a quiescent phase.

In reality, this would hardly be acceptable, as it would imply that a location
executing a long running algorithm would halt the remainder of the network until
it has finished, not to mention the scenario where a location time-locks and thus
halts the entire network of machines.

It would be much more realistic to have time progress in each location
independent of the others. This would isolate the bad effects of time-locks and
introduce more concurrency into the system.

While this would surely have a substantial impact on the semantics, that
currently revolves around the fact that time progresses synchronously, at least
it seems to be a lighter manipulation of the language than solving the name
routing problem.

\vspace{1em}

Another shortcoming of the time model is that it doesn't allow a program to
inspect whether time constraints are satisfied during execution.
This is a problem for programs that need to terminate within a given time
window, as for example signal processing applications often do.

While including the duration of join reactions directly in the time model has
been argued as being a dead end, perhaps a different approach might prove both
sufficient for enabling action on unsatisfied realtime constraints as well as
formally sound.

A potential solution would be to adopt a model of time where the duration of
time instants are considered an external influence on the execution to the
program, and where delayed atoms enter the solution in the time instant that
most closely matches their specified delay.

Consider the following code sketch:
\begin{verbatim}
def
    start<> |> [ series of actions that produce an x<>]
 or x<> & intime<> |> [[ further reductions on x<>, which is still relevant
                      to the problem ]]
 or timeout<> & intime <> |> [[ actions to handle the timeout ]]

 in start<> & intime<> & 10s:timeout<>
\end{verbatim}

This is admittedly speculation however, but could be considered an interesting
future work.

\begin{verbatim}
TODO:
 OK Global name routing
 OK Globally synchronised time
  * Migration security

* potential use cases

* patterns
  - distribution
  - garbage collection
\end{verbatim}


\section{Use Cases}

In this chapter we present our implementation of a simplified version of a
distributed query system applied to a sensor net. Essentially it is just an
instance of the \fixme{insert citation!.. I think its in join-tutorial.}{applet
-server} pattern, although a bit more involved.

The example is heavily inspired by the sensor databases of
\cite{bonnet2001towards}, and evolves around a factory warehouse setting where
each item has a stick-on sensor that measures the temperature. There are also
sensors on the walls and ceilings. All sensors have a unique id available to
them.

Apart from sensors collecting nodes exist, also called \emph{servers}, that
distribute the queries, manifested as join programs, to the sensors. The term
\emph{client} refers to the machine that distributes queries to the collecting
nodes and receives the output of the queries.

\subsection*{Distributed queries in sensor networks}

In \cite{bonnet2001towards}, the concept of sensor databases is introduced, in
which a set of sensors is queried using an extension of SQL, enabling the user
to extract very specific datasets from the network. The queries are executed in
a distributed fashion, so the sensors only transmit data relevant to the query.
For high-resolution sensors generating vast amounts of data, processing the
data locally is cheaper than transmitting it for off-line processing, leading
to a more efficient usage of resources for bandwidth and battery constrained
devices.

The COUGAR project\cite{COUGAR} has already done a lot of research in this
area, and a complete system for executing declarative queries in a distributed
fashion has already been created. However, it would be interesting to see if a
join-calculus based language would prove useful for expressing programs that
essentially do the same as a distributed query. \cite{bonnet2001towards}
mentions that one of the challenges of implementing distributed queries was
handling the asynchronicity of events, which we hope will be easier to deal
with in a join-calculus based language.

\subsection*{Example scenario}

Due to the lack of data structure libraries and the general inefficiency of the
interpreter, we only consider a scenario with one collecting node and two
sensors statically positioned. No new sensors will arrive, but sensor hardware
failure is simulated.

Since our example only features a single collecting node, the server and client
are merged into one machine for simplicity.

The particular query program that is uploaded to the sensors simply measures
the local temperature, simulated by a sinusoidal function, and reports the
average every $n^{\text{th}}$ time instant.  However, if the sensor registers a
temperature above a given threshold it will report it immediately.

Every $m^\text{th}$ time instant the server outputs the received temperature
readings along with an assessment of the age of each reading. If a reading
isn't updated within a given time interval, the program assumes that the sensor
is broken and prompts the operator to replace it.

\subsubsection*{Details of the simulated system}

We will assume a network topology with a central processing node which is
connected to the sensors in a star pattern, using unreliable communication
channels (i.e. some sort of radio protocol).  We do not assume that the runtime
system ensures delivery of messages, but we do assume that any messages that
\emph{do} get delivered are ensured not to be corrupted in any way.

However, since we don't have a mature, compiled language, we will not focus on
memory and processor constraints, since the resource usage depends a lot on the
implementation. We may assume, however, that the actual usage of memory
resources is in some way proportional to the number of active messages and join
patterns on each device , and that the usage of computational resources is
proportional to the number of reduction steps required to arrive at a result.

A built-in name-exchange facility is also simulated, such that the network can
be bootstrapped. This is implemented as a global key-value store that can be
accessed programmatically from any device through the API messages
\verb!register! and \verb!search!. A call to one of these messages always
succeeds in the same time instant. In a real-world scenario, bootstrapping an
ad-hoc network may be more involved, since we cannot rely on a global,
centralized non-failing database.  Most likely one could be moderately
successful in this particular setting by implementing this name-exchange API by
performing regular broadcasts at the \fxnote*{is it called this?}{link-level}
of the network to anyone willing to listen.

The only type of communication that is assumed to be reliable is location
migration, as there is no sensible way to reproduce a location if it is lost
in transit.

\subsection*{Implementation}

Conceptually, a query program is structured in two parts: One that treats the
sensor data inside the sensor itself, and a second part that runs on the server
node and collects and optionally filters the data from the sensors and
eventually aggregate these to the client.

\paragraph{The sensor} runs a minimal program that acts as a shell that first
attempts to register itself with the collecting node and exposes the
sub-location \texttt{query}, that query programs are meant to migrate to.
Figure \ref{fig:sensor-prog} depicts an excerpt of the sensor program that
should reveal this pattern.

During the execution of the sensor program the messages emitted to the server
might be lost in transit because of simulated link breakdown. The effects are
not devastating to this particular communication protocol, because it's a
simple broadcast protocol.  Only in the handshake is it critical to the
execution of the program that delivery is ensured, as the sensor only ever
emits one handshake message.

To address this problem we have also implemented and demonstrated a small code
library, that implements a communication protocol with guaranteed message
delivery above simple join message passing in the next section.

\begin{figure}[!h]
\begin{minipage}{0.97\textwidth}
\begin{verbatim}
def
    [[library definitions]]
 or mkSensor<mscId, handshake> |>
    def
        querysite[
            killQuery<> |> halt<>
         or readTempLoc() |>
              \{return readTemp(mscId) to readTempLoc\}
         in 0
        ]
     in handshake<mscId, readTempLoc, querysite>
 or connect<> |>
      \{ match search("server") with
          Nothing -> { run 1:connect<> }
        | Just handshake ->
            { let mscId = machineId()
            ; run mkSensor<mscId, handshake> \} \}
 in connect<>
\end{verbatim}
\end{minipage}
\caption{The sensor program, with irrelevant parts left out.}\label{fig:sensor-prog}
\end{figure}

\fixme{prepare more for presentation..}

\paragraph{The server} runs a program that accepts sensor handshakes and
constructs the query programs, that migrate to the sensors.  When measurements
arrive from the sensors they are registered and time tagged, and every once in
a while the server prints out the latest measurements, together with their age.
In Figure \ref{fig:server-prog} is an excerpt of the server code without the
parts that deal with low level bookkeeping and library routines.
\begin{figure}[!h]
\begin{minipage}{0.97\textwidth}
\begin{verbatim}
def
    [[library definitions]]
 or mkQuery(threshold, readTemp, callback) |>
    { let num = 10
    ; run def query[
        collect<n, tAkk> |>
          [[ If readTemp() < threshold then
               accumulate temperatures
             else callback(temperature)
           ; if n > num then
               callback(acc. temp./num) ]]
        or migrate<locNm> |>
           { do go(locNm)
           ; let t = readTemp()
           ; run collect<1,t>}
        in 0 ] in {return migrate to mkQuery}
    }
    sensorHandshake<mscId, readT, qLoc> |>
      [[ query = mkQuery(..)
       ; do query(qLoc) ]]
 or sensorCallback<temp>
  & collected<[[collected data]]> |>
      [[ run collected<[[new data]]>
 or printCollected<>
  & collected<data> |>
      { [output data pretty printed]}
      & 5:printCollected<>
      & collected<data>
 in collected<[initial zero-readings]>
    & { do register("server", sensorHandshake) }
    & 5:printCollected<>
\end{verbatim}
\end{minipage}
\caption{The server program, with irrelevant parts left out. Note that code
enclosed in $\left[ \left[\ \right] \right]$ is pseudo code that has an equal
join encoding, which is left out for the sake of readability.}
\label{fig:server-prog}
\end{figure}

\paragraph{The API-calls} used in the code include the \texttt{readTemp} that
reads the current temperature from the simulated sensor hardware.
\texttt{readTemp} is implemented outside the interpreter as a special name.
Also worth noting is \texttt{print}, which also a special name, and the
\texttt{mkClock} name, which is implemented as regular join patterns.

\texttt{mkClock} returns a name, that can be used to query the current time
with a precision of two instants.

\fixme{Should we show the output of the program?}

\subsection*{Discussion}

In constructing this programming example the most noteworthy obstacle, apart
from inexperience in programming join, was the fact that every used library
routine (e.g. list concatenation) had to be included in every sub location in
order to avoid communication penalties in the shape of message loss and time
penalties.


\emph{stuff for generalising} Initially, query programs enter the server node
from a client node through a set of names, that the server node exports.  Then,
as sensors register themselves with the server node, the query programs are
uploaded from the server to the sensors.

In order for this to work, a correct query program is supposed to behave in a
certain way with regards to distribution and internal structure, because
process migration in the join calculus only happens on the initiative of the
process that is to migrate, and cannot be controlled or observed by the
surrounding environment.
----
When a new sensor appears in the network, it enters the registration phase,
where the sensor exports the names used for accessing the sensor hardware, and
the filter-part of the query program migrates to the sensor.  While the sensor
is active and registered, any exchange of sensor data happens only on the
initiative of the filter part of the query program.

Since the nature of queries vary it is up to the query program to determine what
to do if the connection to the server is lost.

\section{A reliable communication protocol}

\fixme{Be more specific with regards to the circumstances of the communication,
or at least refer to what is discussed in the use case, which also needs to be
edited btw.} In order to deal with the imperfections of basic radio and network
communications we have implemented a rudimentary library that enables reliable,
ordered delivery between two communicating parties.

The communication protocol works by the same basic principles as does TCP. In
particular it gives the same guarantees about the message exchanges, namely
those of

\begin{itemize}

\item Retransmission of lost messages

\item Ordered message treatment % currently implemented by assuring that no
%message can be sent before the

\item Duplicate messages are discarded

\end{itemize}

Central to the workings of the protocol is the bookeeping that takes care of
ordering the messages, by labeling each message to be sent with a sequence
number. Only when the recipient acknowledges the reception of a message by
sending an acknowledgement message that carries the same sequence number as the
message of the sender did, the transmission of new messages can continue.  An
overview of the protocol is given below.

\paragraph{The sending side:}
\begin{enumerate}

\item \label{retrans:SYN}
The server sends a \texttt{Syn}-message to initialise the connection

\item If a \texttt{SynAck} message is received within a given time window, the
connection is established and the channel is marked clear, otherwise repeat \ref{retrans:SYN}

\item \label{retrans:Send} When the sender wants to communicate, it increments
the sequence number and send the message along with the sequence number. Then it
removes the clear mark of the channel, effectively blocking the channel.

\item If a \texttt{DataAck} message is received within the given time window,
and its sequence number matches the sender's, the channel is marked clear.
Otherwise the received \texttt{ACK} is discarded, and no further state change
occurs, or if no \texttt{ACK} was received, the message gets retransmitted.

\end{enumerate} \paragraph{The receiving side:} \begin{enumerate}

\item Upon reception of a \texttt{Syn}-message, send a \texttt{SynAck} and set
the sequence number to zero.

\item Upon reception \texttt{Data}-message with higher sequence number,
increment the sequence number and send a \texttt{DataAck}-message containing the
result of computation triggered by the reception if any. Also store the result,
in case the \texttt{DataAck} needs to be retransmitted.

If the sequence number of the \texttt{Data}-message is equal to the currently
recorded sequence number, the last sent \texttt{DataAck} is retransmitted, and
no further computation is triggered.

\end{enumerate}

\paragraph{Shortcomings.} This sketch of a communication protocol is indeed very
basic, and we have made no attempt at refining it further.  This is because we
believe that it is obviously possible to accomplish within the expressive power
of the programming language, looking at TCP for inspiration on optimisations,
and adding more integrity checks to ensure that one buggy participant doesn't
ruin the other.

\subsection{Usage of the library}

Actually using the library as an underlying data communication layer is
accomplished by \emph{proxying} the names used for communication through the
protocol library.  To make this clearer, consider this usage example of a thin
client using server for computation:

\paragraph{Server program}

\begin{verbatim}
def
    include(retransmit.join)
    bigComputation(x) |> {.. return f(x) to bigComputation}
 in
    { let exposedCom = listen(Sync(bigComputation));
    ; register("bigComputation", exposedCom)
    }
\end{verbatim}

\paragraph{Client program}

\begin{verbatim}
def
    include(retransmit.join)
 in
    { let bigCom = search("bigComputation")
    ; match bigCom with
        Just bigCom' -> { let bigComProx = initTrans(Sync, bigCom')
                        ; do print(bigComProx(100))
                        }
        Nothing      -> { do print("Couldn't find bigComputation") }
    }
\end{verbatim}

Above the technique of proxying is applied to yield a two new names,
that each represent an end of the communication channel.
The proxy at the server end is the name returned by the synchronous call to \texttt{listen}. This is the name that is to be published to allow for communication. Feeding this name into the \texttt{initTrans} call yields the final name that the client application interacts with.

Notice that beyond the call to \texttt{initTrans} it is impossible to distinguish a proxied name from an unproxied name, and using the library involves no extra complexity, which is furthers the case that usage of the join calculus shouldn't necessarily be limited to modelling concurrent computation inside a single physical unit, but is also a decent option for a distributed system, even though the communication might be flawed.\fixme{This phrase might be wierd\ldots}
