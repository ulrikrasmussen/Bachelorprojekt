We present here some use cases for a language based on the
distributed join calculus.

Since we don't have a mature, compiled language, we will not focus
too much on memory and processor constraints, since the resource
usage depends a lot on the implementation. We may assume, however,
that the number of active messages and join patterns on each device
is in some way proportional to the actual usage of memory
resources, and that the number of reduction steps required to
arrive at a result is proportional to the usage of computational
resources.

\section{Distributed queries in sensor networks}

In \cite{bonnet2001towards}, the concept of sensor databases is introduced, in
which a set of sensors is queried using an extension of SQL, enabling the user
to extract very specific datasets from the network. The queries are executed in
a distributed fashion, so the sensors only transmit data relevant to the query.
For high-resolution sensors generating vast amounts of data, processing the
data locally is cheaper than transmitting it for offline processing, leading to
a more efficient usage of resources for bandwidth and battery constrained
devices.

The COUGAR project\cite{COUGAR} has already done a lot of research in this
area, and a complete system for executing declarative queries in a distributed
fashion has already been created. However, it would be interesting to see if a
join-calculus based language would prove useful for expressing programs that
essentially does the same as a distributed query. The paper mentioned that one
of the challenges of implementing distributed queries was dealing with the
asynchronicity of events, which we hope will be easy to express in a
join-calculus based language.

\subsection{Example scenario}

The example is heavily inspired by \cite{bonnet2001towards}, and evolves around
a factory warehouse setting where each item has a stick-on sensor that measures
the temperature. There are also sensors on the walls and ceilings. All sensors
has a unique id. By querying the central processing node, it is possible to
retrieve some measure of the position of a given sensor id (determined by
triangulation or proximity to known reference points).

We will try to implement some of the same queries mentioned in the article:

\begin{itemize}
\item
  From all sensors, continuously return all temperatures over a given
  threshold.

\item
  From all sensors on a given floor, continously return the average
  temperature measured over the last minute.

\item
  When two sensors in close proximity to each other in a given time
  window measure a temperature above a given threshold, generate an
  event.

\end{itemize}
\subsection{Assumptions about the underlying system}

We will assume a network topology with a central processing node
which is connected to the sensors in a star pattern, using
unreliable communication channels (i.e. some sort of radio
protocol). The runtime system will assure that messages are
delivered atomically between phyiscal units, as long as they have
an established communication channel. However, sensor nodes may
temporarily come out of contact and become unavailable for extended
periods of time.

We will start out by assuming a static network, where nodes doesn't
move around. The central node knows the names of all its sensors.
The central processing node will always be available, so queries
will get initiated in this by an external client.

\subsection{Doing I/O}

We will begin by defining how the sensors will read data from the
outside world. Since the join-calculus has no way of doing I/O, we
have to define some "magic" messages that forms an API for reading
temperature values.

We read the temperature via explicit probing. We let the API
contain a synchronous message, \verb!read_temperature!, that takes
a continuation and returns a single sample on it as soon as it is
available. This message can be used in the synchronous subset of
the language:

\begin{verbatim}
def loop(t, c) & running<> |>
      {
        run running<>;
        let temp = read_temperature();
        return loop(t+temp, c+1) to loop
      }
 or timeout<> & running<>  |> done<>
 or loop(t, c) & done<>    |> { return (t/c) to loop }
 in {
      run start_timer<10, timeout>;
      run running<>;
      print("Average temperature: " ^ loop(0, 0))
    }
\end{verbatim}
In the example above, we assume that we have string operations as
well as integer and floating point arithmetics. The code repeatedly
reads from the temperature sensor as fast as possible in 10
seconds, and then prints out the average temperature measured.

\subsection{A programmable sensor}

Each sensor is running a minimal program capable of identifying the
sensor with a central node:

\begin{verbatim}
def connect(ping, master) |>
  def here[
           ping_back<> |>
             def pinging<> & ack<> |> start_timer<60, ping_back>
              or pinging<> & timeout<> |> halt<>
              in pinging<> & ping<ack> & start_timer<5, timeout>
        or halt_here<> |> { halt }
        or read_temp_here() |> { return (read_temperature()) to read_temp_here }
        in 0
      ]
   in {
        let success = master(sensor_id(), here, read_temp_here);
        case success of
          True -> { run ping_back<> & fail<here, init> }
        | False -> { run halt_here<> };
        return success to connect
      }

 or init<> |>
      {
        let servers = search();
        match try_connnect(servers) with
          True -> { }
        | False -> { run start_timer<10, init> }
      }

 or try_connect([])   |> { return (False) to try_connect }
 or try_connect((p, m):ms) |> {
       match connect(p, m) with
         True  -> return (True) to try_connect
       | False -> return (try_connect(ms)) to try_connect
    }

 in init<>
\end{verbatim}
The \verb!init! process is the first process to start. This uses an
API message, \verb!search!, to get a list of servers within reach.
Each element in the list is a tuple with two messages defined on
the server: A ping message, and a message for registering the
sensor with the server.

The process \verb!try_connect! attempts to connect to each server,
and stops when it succeeds, returning True.

\verb!connect! defines a new location, \verb!here!, which will
become the root location for all processes that migrates to the
sensor. By using the message \verb!master!, the process tries to
register the sensor with the server, sending the sensor id, the
name of the \verb!here! location and the name of the API message to
read temperature values from the sensor. If the connection
succeeds, a ping process and a fail handler for the location is
started - the ping process halts the location on a timeout, and the
fail handler restarts the \verb!init! process when the location
fails.

Note that in this use case, we exploit the ability to halt
locations to ensure that any processes that may have migrated to
the sensor is killed when we loose connection to the server. We
also "wrap" our exported API calls in messages defined in the
\verb!here! location. This ensures that the exposed interface is
invalidated whenever the location is killed, ensuring that no
process can continue reading from the sensor from an external
location (which would destroy the purpose of migrating queries to
the sensors in the first place).

\subsection{Programming the central nodes}

Each central node is running an instance of the following program:

\begin{verbatim}
def register(id, loc, read_temp) & sensors<xs> |>
    {
      run sensors<(id, loc, read_temp):xs>;
      return (True) to register
    }
 or ping() |> return () to ping

 or runQuery(constructor) & sensors<xs> |>
    sensors<xs>
    & def here[
            in {
                 let mkQ = constructor(here);

               }
          ]
       in ...
\end{verbatim}
\subsection{Querying sensors}

(This part is not finished yet, and hasn't been thought through). A
query is a program that takes the following form:

\begin{verbatim}
def mkCollector(cloc) |>
    def collector[
          mkSensorQuery(sloc, read_temp) |>
              def sensorQuery[
                     T in { go(sloc); ... }
                  ]
               in 0
          in { go(cloc); ... }
        ]
 in { return (mkSensorQuery) to mkCollector }
\end{verbatim}
\verb!mkCollector! instantiates a process that will receive data
from all of the sensors, combine it in some way and return the
result to the user. We also define a \verb!collector! location that
migrates to a given location (this would be a location on one of
the central nodes). \verb!mkCollector! returns a message that
instantiates a process to be executed on each sensor, given a
location on the sensor to migrate to, and an interface for reading
data.

