We present here some use cases for a language based on the
distributed join calculus.

Since we don't have a mature, compiled language, we will not focus
too much on memory and processor constraints, since the resource
usage depends a lot on the implementation. We may assume, however,
that the number of active messages and join patterns on each device
is in some way proportional to the actual usage of memory
resources, and that the number of reduction steps required to
arrive at a result is proportional to the usage of computational
resources.


\section{Distributed queries in sensor networks}

In \cite{bonnet2001towards}, the concept of sensor databases is
introduced, in which a set of sensors is queried using an extension of
SQL, enabling the user to extract very specific datasets from the
network. The queries are executed in a distributed fashion, so the
sensors only transmit data relevant to the query.  For high-resolution
sensors generating vast amounts of data, processing the data locally
is cheaper than transmitting it for offline processing, leading to a
more efficient usage of resources for bandwidth and battery
constrained devices.

The COUGAR project\cite{COUGAR} has already done a lot of research in
this area, and a complete system for executing declarative queries in
a distributed fashion has already been created. However, it would be
interesting to see if a join-calculus based language would prove
useful for expressing programs that essentially does the same as a
distributed query. The paper mentioned that one of the challenges of
implementing distributed queries was dealing with the asynchronicity
of events, which we hope will be easy to express in a join-calculus
based language.


\subsection{Example scenario}

The example is heavily inspired by \cite{bonnet2001towards}, and
evolves around a factory warehouse setting where each item has a
stick-on sensor that measures the temperature. There are also
sensors on the walls and ceilings. All sensors has a unique id. By
querying the central processing node, it is possible to retrieve
some measure of the position of a given sensor id (determined by
triangulation or proximity to known reference points).

We will try to implement some of the same queries mentioned in the
article:

\begin{itemize}
\item
  From all the sensors, continuously return all temperatures over a
  given threshold.

\item
  From all sensors on a given floor, continously return the average
  temperature measured over the last minute.

\item
  When two sensors in close proximity to each other in a given time
  window measure a temperature above a given threshold, generate an
  event.
\end{itemize}


\subsection{Assumptions about the underlying system}

We will assume a network topology with a central processing node
which is connected to the sensors in a star pattern, using
unreliable communication channels (i.e. some sort of radio protocol).  
We do not assume that the runtime system ensures delivery of messages,
but we do assume that any message that \emph{does} get delivered are
ensured not to be corrupted in any way.

We will start out by assuming a static network, where nodes doesn't
move around. A central node will always be present in the system, and
it will be from this node that all queries are initiated.

It is also assumed that there is a built in name-exchange facility,
such that the network can be bootstrapped. In our scenario, this is a
global key-value store that can be accessed programmatically from any
device through the API messages \verb!register! and \verb!search!. Any
call to one of these messages always succeeds. In a real-world
scenario, 


\subsection{Implementation}

\subsubsection{Communication protocols and code conventions}

Initially, query programs enter the server node from a client node
through a set of names, that the server node exports.
Then, as sensors register themselves with the server node, the query
programs are uploaded from the server to the sensors.

In order for this to work, a correct query program is supposed to 
behave in a certain way with regards to distribution and internal
structure, because process migration in the join calculus only happens on 
the initiative of the process that is to migrate, and cannot be controlled
or observed by the surrounding environment.
Conceptually, a query program is structured in two parts: One that 
treats the sensor data inside the sensor itself, and a second part
that runs on the server node that collects and filters the data from
the sensors and eventually aggregate these to the client. We will refer
to these two parts as the \emph{reducer} and \emph{filter} parts, respectively.
The interface of the query programs is more explicitly specified later
in this section.

When a new sensor appears in the network, it enters the registration
phase, where the sensor exports the names used for accessing the 
sensor hardware, and the filter-part of the query program migrates to
the sensor.
While the sensor is active and registered, any exchange of sensor data
happens only on the initiative of the filter part of the query
program.

When the communication between a sensor and the server times out, the
default behaviour is to reboot the sensor back into the registration
phase.
However, depending on the type of query being run, this may not be desirable,
especially for long running queries.\fixme{Decide on a solution!}
One way around this would be to allow the query program to determine
the length of the timespan before a timeout, while still eventually 
having the sensor reboot when the timeout is reached.

Since everything is asynchronous, the server and the sensor may each
have a slightly different idea of just when a timeout occurs, and it
is possible that the sensor notices a timeout before the server and
therefore attempts to register again, even though the server still
considers the sensor as registered. In this case the server should
just re-register the sensor and delete its old reference.\fixme{ Also
treat the case where the server times out before the sensor, even
though it seems improbable (impossible)}

\subsubsection{Doing I/O}

We will begin by defining how the sensors will read data from the
outside world. Since the join-calculus has no way of doing I/O, we
have to define some ``magic'' messages that forms an API for reading
temperature values.

We read the temperature via explicit probing. We let the API
contain a synchronous message, \verb!read_temperature!, that takes
a continuation and returns a single sample on it as soon as it is
available. This message can be used in the synchronous subset of
the language:

\begin{verbatim}
def loop(t, c) & running<> |>
      {
        run running<>;
        let temp = read_temperature();
        return loop(t+temp, c+1) to loop
      }
 or timeout<> & running<>  |> done<>
 or loop(t, c) & done<>    |> { return (t/c) to loop }
 in {
      run start_timer<10, timeout>;
      run running<>;
      print("Average temperature: " ^ loop(0, 0))
    }
\end{verbatim}

In the example above, we assume that we have string operations as
well as integer and floating point arithmetics. The code repeatedly
reads from the temperature sensor as fast as possible in 10
seconds, and then prints out the average temperature measured.

\subsubsection{A programmable sensor}

We need to run a static program on each sensor, which forms a framework
for the query programs to use.
Each sensor is running a minimal program capable of identifying the
sensor with the central node, and continuously verifies the
connectivity with the central node using a ping process.

\begin{verbatim}
def connect(ping, register) |>
  def here[
           ping_back<> |>
             def pinging<> & ack<> |>
                    start_timer<60, ping_back>
              or pinging<> & timeout<> |>
                    halt<>
              in pinging<>
               & ping<ack>
               & start_timer<5, timeout>

        or halt_here<> |> { halt }
        or read_temp_here() |>
           {
             return (read_temperature())
                    to read_temp_here
           }
        in 0
      ]
   in {
        let success = register(sensor_id(),
                               here,
                               read_temp_here);
        case success of
          True -> { run ping_back<> & fail<here, init> }
        | False -> { run halt_here<> };
        return success to connect
      }

 or init<> |>
      {
        let servers = search();
        match try_connnect(servers) with
          True -> { }
        | False -> { run start_timer<10, init> }
      }

 or try_connect([])   |>
    {
       return (False) to try_connect
    }
 or try_connect((p, r):ms) |> {
       match connect(p, r) with
         True  -> { return (True)
                           to try_connect }
       | False -> { return (try_connect(ms))
                           to try_connect }
    }

 in init<>
\end{verbatim}

The \verb!init! process is the first process to start. This uses an
API message, \verb!search!, to get a list of servers within reach.
Each element in the list is a tuple with two messages defined on
the server: A ping message, and a message for registering the
sensor with the server.

The process \verb!try_connect! attempts to connect to each server,
and stops when it succeeds, returning True.

\verb!connect! defines a new location, \verb!here!, which will
become the root location for all processes that migrates to the
sensor. By using the message \verb!master!, the process tries to
register the sensor with the server, sending the sensor id, the
name of the \verb!here! location and the name of the API message to
read temperature values from the sensor. If the connection
succeeds, a ping process and a fail handler for the location is
started - the ping process halts the location on a timeout, and the
fail handler restarts the \verb!init! process when the location
fails.

Note that in this use case, we exploit the ability to halt
locations to ensure that any processes that may have migrated to
the sensor is killed when we loose connection to the server. We
also "wrap" our exported API calls in messages defined in the
\verb!here! location. This ensures that the exposed interface is
invalidated whenever the location is killed, ensuring that no
process can continue reading from the sensor from an external
location (which would destroy the purpose of migrating queries to
the sensors in the first place).

\subsubsection{Programming the central nodes}

Each central node is running an instance of the following program:

\begin{verbatim}
def register(id, loc, read_temp) & sensors<xs> |>
    {
      run sensors<(id, loc, read_temp):xs>;
      return (True) to register
    }
 or ping() |> return () to ping

 or runQuery(constructor) & sensors<xs> |>
    sensors<xs>
    & def here[
            in {
                 let mkQ = constructor(here);

               }
          ]
       in ...
\end{verbatim}


\subsection{Querying sensors}
\label{querying_sensors}

All queries are programs which must implement a certain interface to be
compatible with our sensor framework. We will call such a program a well-formed
query program. An example of such a program (which does nothing) is seen below:

\begin{verbatim}
def mkReducer(cloc) |>
    def reducer[
          mkFilter(sloc, read_temp) |>
              def filter[
                     T in { go(sloc); ... }
                  ]
               in { return () to mkSensorQuery }
          in { go(cloc); ... }
        ]
     in { return (mkFilter) to mkReducer }
 in ...
\end{verbatim}

The client sends the the synchronous message \verb!mkReducer! message
to the central node, which takes care of distributing the program to all
registered sensors in the following way:
\begin{enumerate}
 \item
   The boot program on the central node calls \verb!mkReducer!, sending the name
   of a location that resides on the central node with it.
 \item
   This instantiates a new location, \verb!reducer!, which migrates to the
   location given to \verb!mkReducer! (physically moving the \verb!reducer! to
   the central node).
 \item
   Inside the \verb!reducer! location, the name \verb!mkFilter! is defined.
   This name is sent back to the caller of \verb!mkReducer! (the boot program on
   the central node).
 \item
   The boot program has a list of all the registered sensors. For each sensor,
   the boot program calls \verb!mkFilter!, sending with it the name of a
   location on the sensor and the name of a synchronous message for reading
   temperature values on the sensor.
 \item
   This instantiates a new location, \verb!filter! which migrates to the
   location given to \verb!mkFilter! (phyiscally moving instances of
   \verb!filter! to each sensor).
\end{enumerate}

When this process is done, the query program is deployed. Each of the
\verb!filter! locations will read values from its respective sensors, and send
messages back to the \verb!reducer! location which combines the data. Since
the \verb!reducer! location is defined in the scope of the client program,
it can send back results to the client using message names obtained from the
scope.
