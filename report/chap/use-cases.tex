% vim:spell:spelllang=en_gb:
% vim:set textwidth=80:

\section{Use Cases}

In this chapter we present our implementation of a simplified version of a
distributed query system applied to a sensor net. Essentially it is just an
instance of the \fixme{insert citation!.. I think its in join-tutorial.}{applet
-server} pattern, although a bit more involved.

The example is heavily inspired by the sensor databases of
\cite{bonnet2001towards}, and evolves around a factory warehouse setting where
each item has a stick-on sensor that measures the temperature. There are also
sensors on the walls and ceilings. All sensors have a unique id available to
them.

Apart from sensors collecting nodes exist, also called \emph{servers}, that
distribute the queries, manifested as join programs, to the sensors. The term
\emph{client} refers to the machine that distributes queries to the collecting
nodes and receives the output of the queries.

\subsection*{Distributed queries in sensor networks}

In \cite{bonnet2001towards}, the concept of sensor databases is introduced, in
which a set of sensors is queried using an extension of SQL, enabling the user
to extract very specific datasets from the network. The queries are executed in
a distributed fashion, so the sensors only transmit data relevant to the query.
For high-resolution sensors generating vast amounts of data, processing the
data locally is cheaper than transmitting it for off-line processing, leading
to a more efficient usage of resources for bandwidth and battery constrained
devices.

The COUGAR project\cite{COUGAR} has already done a lot of research in this
area, and a complete system for executing declarative queries in a distributed
fashion has already been created. However, it would be interesting to see if a
join-calculus based language would prove useful for expressing programs that
essentially do the same as a distributed query. \cite{bonnet2001towards}
mentions that one of the challenges of implementing distributed queries was
handling the asynchronicity of events, which we hope will be easier to deal
with in a join-calculus based language.

\subsection*{Example scenario}

Due to the lack of data structure libraries and the general inefficiency of the
interpreter, we only consider a scenario with one collecting node and two
sensors statically positioned. No new sensors will arrive, but sensor hardware
failure is simulated.

Since our example only features a single collecting node, the server and client
are merged into one machine for simplicity.

The particular query program that is uploaded to the sensors simply measures
the local temperature, simulated by a sinusoidal function, and reports the
average every $n^{\text{th}}$ time instant.  However, if the sensor registers a
temperature above a given threshold it will report it immediately.

Every $m^\text{th}$ time instant the server outputs the received temperature
readings along with an assessment of the age of each reading. If a reading
isn't updated within a given time interval, the program assumes that the sensor
is broken and prompts the operator to replace it.

\subsubsection*{Details of the simulated system}

We will assume a network topology with a central processing node which is
connected to the sensors in a star pattern, using unreliable communication
channels (i.e. some sort of radio protocol).  We do not assume that the runtime
system ensures delivery of messages, but we do assume that any messages that
\emph{do} get delivered are ensured not to be corrupted in any way.

However, since we don't have a mature, compiled language, we will not focus on
memory and processor constraints, since the resource usage depends a lot on the
implementation. We may assume, however, that the actual usage of memory
resources is in some way proportional to the number of active messages and join
patterns on each device , and that the usage of computational resources is
proportional to the number of reduction steps required to arrive at a result.

A built-in name-exchange facility is also simulated, such that the network can
be bootstrapped. This is implemented as a global key-value store that can be
accessed programmatically from any device through the API messages
\verb!register! and \verb!search!. A call to one of these messages always
succeeds in the same time instant. In a real-world scenario, bootstrapping an
ad-hoc network may be more involved, since we cannot rely on a global,
centralized non-failing database.  Most likely one could be moderately
successful in this particular setting by implementing this name-exchange API by
performing regular broadcasts at the \fxnote*{is it called this?}{link-level}
of the network to anyone willing to listen.

The only type of communication that is assumed to be reliable is location
migration, as there is no sensible way to reproduce a location if it is lost
in transit.

\subsection*{Implementation}

Conceptually, a query program is structured in two parts: One that treats the
sensor data inside the sensor itself, and a second part that runs on the server
node and collects and optionally filters the data from the sensors and
eventually aggregate these to the client.

\paragraph{The sensor} runs a minimal program that acts as a shell that first
attempts to register itself with the collecting node and exposes the
sub-location \texttt{query}, that query programs are meant to migrate to.
Figure \ref{fig:sensor-prog} depicts an excerpt of the sensor program that
should reveal this pattern.

During the execution of the sensor program the messages emitted to the server
might be lost in transit because of simulated link breakdown. The effects are
not devastating to this particular communication protocol, because it's a
simple broadcast protocol.  Only in the handshake is it critical to the
execution of the program that delivery is ensured, as the sensor only ever
emits one handshake message.

To address this problem we have also implemented and demonstrated a small code
library, that implements a communication protocol with guaranteed message
delivery above simple join message passing in the next section.

\begin{figure}[!h]
\begin{minipage}{0.97\textwidth}
\begin{verbatim}
def
    [[library definitions]]
 or mkSensor<mscId, handshake> |>
    def
        querysite[
            killQuery<> |> halt<>
         or readTempLoc() |>
              \{return readTemp(mscId) to readTempLoc\}
         in 0
        ]
     in handshake<mscId, readTempLoc, querysite>
 or connect<> |>
      \{ match search("server") with
          Nothing -> { run 1:connect<> }
        | Just handshake ->
            { let mscId = machineId()
            ; run mkSensor<mscId, handshake> \} \}
 in connect<>
\end{verbatim}
\end{minipage}
\caption{The sensor program, with irrelevant parts left out.}\label{fig:sensor-prog}
\end{figure}

\fixme{prepare more for presentation..}

\paragraph{The server} runs a program that accepts sensor handshakes and
constructs the query programs, that migrate to the sensors.  When measurements
arrive from the sensors they are registered and time tagged, and every once in
a while the server prints out the latest measurements, together with their age.
In Figure \ref{fig:server-prog} is an excerpt of the server code without the
parts that deal with low level bookkeeping and library routines.
\begin{figure}[!h]
\begin{minipage}{0.97\textwidth}
\begin{verbatim}
def
    [[library definitions]]
 or mkQuery(threshold, readTemp, callback) |>
    { let num = 10
    ; run def query[
        collect<n, tAkk> |>
          [[ If readTemp() < threshold then
               accumulate temperatures
             else callback(temperature)
           ; if n > num then
               callback(acc. temp./num) ]]
        or migrate<locNm> |>
           { do go(locNm)
           ; let t = readTemp()
           ; run collect<1,t>}
        in 0 ] in {return migrate to mkQuery}
    }
    sensorHandshake<mscId, readT, qLoc> |>
      [[ query = mkQuery(..)
       ; do query(qLoc) ]]
 or sensorCallback<temp>
  & collected<[[collected data]]> |>
      [[ run collected<[[new data]]>
 or printCollected<>
  & collected<data> |>
      { [output data pretty printed]}
      & 5:printCollected<>
      & collected<data>
 in collected<[initial zero-readings]>
    & { do register("server", sensorHandshake) }
    & 5:printCollected<>
\end{verbatim}
\end{minipage}
\caption{The server program, with irrelevant parts left out. Note that code
enclosed in $\left[ \left[\ \right] \right]$ is pseudo code that has an equal
join encoding, which is left out for the sake of readability.}
\label{fig:server-prog}
\end{figure}

\paragraph{The API-calls} used in the code include the \texttt{readTemp} that
reads the current temperature from the simulated sensor hardware.
\texttt{readTemp} is implemented outside the interpreter as a special name.
Also worth noting is \texttt{print}, which also a special name, and the
\texttt{mkClock} name, which is implemented as regular join patterns.

\texttt{mkClock} returns a name, that can be used to query the current time
with a precision of two instants.

\fixme{Should we show the output of the program?}

\subsection*{Discussion}

In constructing this programming example the most noteworthy obstacle, apart
from inexperience in programming join, was the fact that every used library
routine (e.g. list concatenation) had to be included in every sub location in
order to avoid communication penalties in the shape of message loss and time
penalties.

A potential problem with the program is revealed in the event of a query with
devious behaviour entering the system, with the devious behaviour consisting in
for example neglecting to migrate to the sensors after construction, resulting
in a significant increase in data traffic between the sensors and the servers,
thus increasing battery drain.

Unfortunately there's no cure for this, because there is no way in the join
program to detect if the definition site of a name currently resides on the same
physical machine as the sender.

---

With a little more work the example could be extended to support the dynamic arrival of new sensors, and with further work the dynamic reconfiguration of sensors

\emph{stuff for generalising} Initially, query programs enter the server node
from a client node through a set of names, that the server node exports.  Then,
as sensors register themselves with the server node, the query programs are
uploaded from the server to the sensors.

In order for this to work, a correct query program is supposed to behave in a
certain way with regards to distribution and internal structure, because
process migration in the join calculus only happens on the initiative of the
process that is to migrate, and cannot be controlled or observed by the
surrounding environment.
----
When a new sensor appears in the network, it enters the registration phase,
where the sensor exports the names used for accessing the sensor hardware, and
the filter-part of the query program migrates to the sensor.  While the sensor
is active and registered, any exchange of sensor data happens only on the
initiative of the filter part of the query program.

Since the nature of queries vary it is up to the query program to determine what
to do if the connection to the server is lost.

