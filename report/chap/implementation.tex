% vim:spell:spelllang=en_gb:
% vim:set textwidth=80 fo+=tcroql:

%Introduction

In order to be able to empirically test the sample programs we have devised, we
have constructed a proof-of-concept implementation of a simulator and
interpreter, that is capable of executing programs written in Join and simulate
external events such as link downtimes and hardware sensors and actuators.

The simulator is implemented in Haskell because it provided a nice foundation in
the shape of a combinatoric parser library and a very extensive type system. The
parser library called \texttt{Parsec} \cite{parsec} facilitated a rather
concise mapping of the concrete syntax into abstract syntax, and by defining
appropriate monads and type classes we feel that we have an implementation,
that doesn't contain too much clutter and plumbing.

\section{Overview}
The interpreter consists of two major parts: One that handles the communication
between machines and the progress of time, and one that performs reductions
inside a single machine. These two parts are called \texttt{GlobalInterpreter}
and \emph{Interpreter} respectively.

Beyond communication and time \texttt{GlobalInterpreter} also manages the
\emph{EventLog}, which is a list containing information about the values
registered by hardware sensors in given time instants as well as machine link
indicators. Migration is also managed by the global interpreter.

Because our examples contain only spatially inert machines, the only actuators
modelled have been print-commands.

The global interpreter runs a loop in a fashion somewhat similar to what the
semantics prescribe:

\begin{enumerate}
\item Match calls to sensors and actuators in each machine
\item Spawn new sublocations
\item Migrate locations
\item Communicate
\item Step individual interpreters
\end{enumerate}

This deviates slightly from what the rules in the semantics prescribe, but we
consider the two to be largely equivalent. The chief reason for the difference
is that the interpreter was put together bit by bit and used as a playground for
new extensions, of which not all were deemed important enough to make it to the
semantics. However, having now a concrete directed semantics for the programming
language would probably greatly assist a potential new reimplementation without
all the bits left over from obsolete extensions.

\section{Implementation details}
The simulator consists in its entirety of the following files, each comprising a
Haskell module:
\begin{description}

\item[Language.hs] The Language module defines the entire abstract syntax along
with the basic variable analysis functions defined on it, such as
\emph{freevars}, \emph{definedvars}, \emph{substitution} etc.

\item[Parser.hs] The Parser module makes out the syntactical parser. It
transforms the concrete syntax of the input programs to abstract syntax.

\item[Desugar.hs] The desugariser converts the sugared parts of the abstract
syntax of a program to the equivalent unsugared representation. Henceforth it is
an invariant of the system that only the unsugared parts of the abstract syntax
is present in the syntax tree.

\item[Interpreter.hs] The Interpreter module performs reductions and heating on
a single machine, and also manages local garbage collection.

\item[GlobalInterpreter.hs] The GlobalInterpreter module facilitates the
exchange of messages between the simulated machines, based on the
\emph{EventLog} of the simulation. It also feeds sensor data to the machines in
the network and collects actuator data.

\item[Aux.hs] The auxiliary module contains functions and datatypes for starting
the simulator. Executing a sample Join program typically involves a call to the
function \texttt{stdJoinMain} along with parameters controlling the simulation.

\item[JoinApi.hs] The JoinApi module contains definitions of functions present
implicitly in every join machine. These are mostly arithmetic and boolean
operations, but also the means for name exchange.

\end{description}

\subsection{The EventLog}

The event log comprises a list of all the external events that influence the
system at any given time interval. These events are specified via helper
functions and a concatenation operator \texttt{+\&+}.

Consider this example, taken from the use case:
\begin{verbatim}
events = ([
           EvLinkUp "Server" "Sensor_A"
         , EvLinkUp "Server" "Sensor_B"
         ]:(cycle [[]]))
         +&+ (atTime 20 $ linkUpProb 42 "Server" "Sensor_B" 0.6)
         +&+ (mkSpecial 1 "readTemp" readTemp)
         +&+ (mkSpecial 1 "print" jPrint)
\end{verbatim}

This example specifies that the server communicates without trouble with both
sensors for the first twenty time instants. From time 20 and thereafter the link
between the server and Sensor\_B alternates between up and down with a $60\%$
chance of being up.
The \texttt{mkSpecial} functions creates the sensor \texttt{readTemp} and the
actuator \texttt{print}.

\subsection{Garbage collection}
The garbage collection scheme applied closely resembles the well known
\emph{Mark-Sweep} algorithm, and works as follows, initially disregarding the
distribution extension:
\begin{enumerate}
\item Mark all atoms in the solution as live.
\item Mark all matchable defs as live, add their producible atoms live,
recursing into nested defs.\label{gc:markdef}
\item If the live set was expanded, go to \ref{gc:markdef}.
\item Replace the sets of defs with the subsets marked live.
\end{enumerate}

However, introducing distribution one needs to maintain a list of live names that
escape their defining machine, as it is impossible to
conclude on a local basis, that the name is unproducible in the global context.

While this might initially seem to be a big drawback, in most of the practically
observed situations it could be handled quite easily by either isolating
dynamically created functionality, with escaping names, in a location which
could be halted when the communication protocols employed by the program would
deem them irrelevant.  Alternatively one could add an extra atom to the join
patterns defining the escaping name, which could be consumed when the name
became irrelevant, resulting in the inability to mark the implicated defs live.

Consider this example. Names are assumed to be exported already.
\begin{verbatim}
Machine1:            | Machine2:
def                  | def
    x<> |> .. y<> .. |     y<> |> ...
    z<> |> .. x<> .. |  in 0
 in z<>              |
\end{verbatim}

In the above example there is no way for machine two to conclude that its def
could be garbage collected, because the name $\atm y<>$ has escaped to machine
one, and only when it is done using it can the def be collected.

However, if the required lifetime of $\atm y<>$ is known beforehand by the
programmer, guarding the def with an extra atom allows it to be collected, as
shown below:
\begin{verbatim}
Machine1:            | Machine2:
def                  | def
    x<> |> .. y<> .. |     y<> & alive<> |> alive<> & ...
    z<> |> .. x<> .. |     done<> & alive<> |> 0
 in z<>              |  in alive<>
\end{verbatim}

Upon the issuing of the $\atm done<>$ atom the $\atm alive<>$ atom will be
consumed and the $\atm y<>$-def can be safely garbage collected.

\subsection{Constructing an experiment}
The auxiliary module which contains functions for
starting the interpreter and datatypes for parametrising it. The central piece
of the library is the function \texttt{stdJoinMain}, which starts the simulator
with the given parameters, which are:

\begin{enumerate}

\item A map that maps atoms to built in function. This is usually used to make
low level arithmetic operations available to join programs. In the code this is
termed the \emph{ApiMap}.

\item A complete list of the physical machines that are present, given as list
of pairs of machine class and machine name.

\item A specification of what kinds of machines should be present in the
network, along with the files that contains their initial programs. This is
termed \emph{Machine Classes} in the code.

\item A data record containing parameters to the interpreter, such as
termination conditions.

\item An infinite list of external events, usually constructed by helper
functions.

\end{enumerate}

As an example, consider the code below that starts up a simple network of two
machines:
\begin{verbatim}
module Main(main) where

import Aux
import Language
import Interpreter

mClasses = [ ("Simple1", "Simple1.join")
           , ("Simple2", "Simple2.join")]

machines = [ ("Machine_1", "Simple1")
           , ("Machine_2", "Simple2")]

api = []

jPrint t (MsgA _ [jStr, VarE k]) =
  ([OutMessage t (fromJoin jStr)],[MsgA k []])

events = (mkSpecial 1 "print" jPrint)
         +&+ ([EvLinkUp "Machine_1" "Machine_2"]:(cycle [[]]))

main = stdJoinMain api machines mClasses
         defaultConfig{breakAtTime = Just 100} events
\end{verbatim}

The featured machines are \texttt{Machine\_1} and \texttt{Machine\_2}, which are
seeded with the Join programs residing in the files \texttt{Simple1.join} and
\texttt{Simple2.join}. Initially they are able to communicate, and the
simulation terminates when it has reached time $100$.

