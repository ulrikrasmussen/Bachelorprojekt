% vim:spell:spelllang=en_gb:
% vim:set textwidth=80 fo+=tcroql:

%Introduction

In order to be able to empirically test the constructed sample programs we have
constructed a proof-of-concept implementation of an interpreter/simulator that
is capable of executing programs written in Join and simulate external events
such as link downtimes and hardware functions.

The simulator is implemented in Haskell because it provided a nice foundation in
the shape of a combinatoric parser library and a very expansive type system. The
parser library, \texttt{Parsec}, facilitated a rather concise mapping of the
concrete syntax into abstract syntax, and by defining appropriate monads and
type classes we feel that we have at an implementation, that doesn't contain too
much clutter and plumbing.\fixme{Should we actually include this?}

\section{Features and usage}
The simulator is capable of simulating a network of physical machines, each
executing a join program and providing a set of built in join patterns that make
it up for hardware functions.

It is possible to indicate that a link between two machines in the network
toggles its state between being up and down as a function of the time, thereby providing a means for reasoning about the delivery of messages.

However, the simulator interprets the truth of the $\diamond$-relation to be
given by reachability between $a$ and $b$ in the graph consisting of all
machines which have established links between each other, thereby simulating
that nodes will actively route messages between otherwise disconnected senders
and receivers.

In a real implementation the exact behaviour in this aspect would probably be
more or less determined by the targeted domain of the implementation. For
programs running across machines connected to the internet either total absence
or presence of communication would likely be assumed.

The mechanism that enables the indication of link status is called the
\emph{EventLog}, and it also provides a means to introduce special unary join
patterns that produce a given result as a function of the current time.

\section{Structure}

The simulator is structured in basically the same overall components as the
semantics that it strives to implement. However, as the implementation is in
practise subject to the limitations of computer hardware it also includes a
garbage collector.

The simulator consists of three major parts:
\begin{itemize}
\item A Library of auxiliary functions for starting the simulator.
\item An outer interpreter that
      \begin{itemize}
      \item invokes the inner interpreters in sequence,
      \item contains a subsystem for handling communication and
      \item contains a subsystem for migration and halting.
      \end{itemize}
\item An inner interpreter that
      \begin{itemize}
      \item transform atoms to their normal form (i.e. heats defs etc.),
      \item match join patterns and perform reactions and
      \item collect garbage.
      \end{itemize}
\end{itemize}

The garbage collection scheme applied closely resembles the well known
\emph{Mark-Sweep} algorithm, and works as follows, initially disregarding the
distribution extension:
\begin{enumerate}
\item Mark all atoms in the solution as live.
\item Mark all matchable defs as live, add their producible atoms live,
recursing into nested defs.\label{gc:markdef}
\item If the live set was expanded, go to \ref{gc:markdef}.
\item Replace the sets of defs and atoms with the subsets marked live.
\end{enumerate}

However, introducing distribution one needs to maintain a list of names that
escape their defining machine marking them live, as it is impossible to
conclude on a local basis, that the name is unproducible in the global context.

While this might initially seem to be a big drawback, in most of the practically
observed situations it could be handled quite easily by either isolating
dynamically created functionality, with escaping names, in a location which
could be halted when the communication protocols employed by the program would
deem them irrelevant.  Alternatively one could add an extra atom to the join
patterns defining the escaping name, which could be consumed when the name
became irrelevant, resulting in the inability to mark the implicated defs live.

\subsection*{Deviations from the semantics}
\fixme{Find out what to actually change in the interpreter and where the code
deviates}
Since the interpreter was devised prior to the operational semantics as a
playground or sandbox, that served both the purpose of executing sample programs
and the purpose of providing us with an intuition of the workings of the
calculus, there are some differences between the semantics and the actual
implementation.

\section{Constructing an experiment}
Invoking the simulator is typically done by creating a haskell module featuring
\begin{itemize}

\item A specification of what kinds of machines should be present in the
network, along with the files that contains their initial programs. This is
termed \emph{Machine Classes} in the code.

\item A complete list of the physical machines that are present, given as list
of pairs of machine class and machine name.

\item An infinite list of external events, usually constructed by helper
functions.

\item A data record containing parameters to the interpreter, such as
termination conditions.

\item A map that maps atoms to built in function. This is usually used to make
low level arithmetic operations available to join programs. In the code this is
termed the \emph{ApiMap}.

\end{itemize}
