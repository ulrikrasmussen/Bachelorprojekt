% vim:spell:spelllang=en_gb:
% vim:set textwidth=80 fo+=tcroql:

\section{General discussion}

While we have been able to successfully produce sample programs without the need
to resolve to using cumbersome encodings, and demonstrated the rather
transparent way the proxying pattern could be applied to deal satisfactorily
with the issues of unreliable communication, some fundamental implementation
issues need to be solved before the use of the join calculus as a foundation for
distributed programming can be said to be realistic.

\subsection*{Migration problems}
When a location migrates, every location it has ever exported a name to knows
which physical machine to send new messages to.  Since the definition site of a
name has no given means to know the extent that a name escapes into the network
once it has escaped its scope, it is impossible to notify every site that is
knowledgeable of the name that its definition site has moved to a different
physical machine.

The following code exemplifies the problem:
\begin{verbatim}
def
    get_trig<k> & app_trig<t> |> k<t>
 or server[
       trigger<t> |> t<> & 10:trigger<t>
    in get_trig<trigger>
    ]
 or applet[
       do_work<> |> ...
    or trigger<> |> ...
    in { run app_trigger<trigger>
       ; do go(client_1)
       ; run do_work<>
       ; do go(client_2)
       ; ...
       ; do go(client_n)}
    ]
 or site_1[...] or ... or site_n[...]
\end{verbatim}

If we assume that the locations \verb+site_1...site_n+ represent different
physical machines, some special means is needed to route the trigger messages to
the applet as it migrates.

\vspace{1em}

If one were to speculate in possible solutions to this problem one could imagine
either a central name server or a distributed key-value store that maps names to
machines. This seems horribly ineffective, because it seems that a consequence
is that everybody needs to know where everybody else is, and either produce a
single point of failure or a lot of network traffic containing administrative
overhead whenever locations migrate.

\vspace{1em}

An entirely different approach would be to associate a name to the physical
machine that hosted the escaping context, and require the migrating locations to
implement a scheme manually that solves the problem satisfactorily in the given
scenario. This might possibly work if the actual communication requirements in
most applications are easier to handle than the general name routing problem.
However, further investigation of this is beyond the scope of this thesis, but
could be considered an interesting future work.

\subsection*{Synchronised time}

Another unrealistic aspect of the language is the model of time that it
exhibits. In the current model time progresses only when every machine has
finished its time instant, such that it is in a quiescent phase.

In reality, this would hardly be acceptable, as it would imply that a location
executing a long running algorithm would halt the remainder of the network until
it has finished, not to mention the scenario where a location time-locks and thus
halts the entire network of machines.

It would be much more realistic to have time progress in each location
independent of the others. This would isolate the bad effects of time-locks and
introduce more concurrency into the system.

\vspace{1em}

Another shortcoming of the time model is that it doesn't allow a program to
inspect whether time constraints are satisfied during execution.
This is a problem for programs that need to terminate within a given time
window, as for example signal processing applications often do.

While including the duration of join reactions directly in the time model has
been argued as being a dead end, perhaps a different approach might prove both
sufficient for enabling action on unsatisfied realtime constraints as well as
formally sound.

%A potential solution would be to adopt a model of time where the duration of
%time instants are considered an external influence on the execution of the
%program, and where delayed atoms enter the solution in the time instant that
%most closely matches their specified delay.
%
%Consider the following code sketch:
%\begin{verbatim}
%def
%    start<> |> [ instant that produces an x<>]
% or x<> & intime<> |>
%         [[ further reductions on x<>,
%            which is still relevant to the problem ]]
% or !timeout<> & intime <> |>
%         [[ actions to handle the timeout ]]
%
% in start<> & 1:intime<> & 10s:timeout<>
%\end{verbatim}
%
%This is admittedly speculation however, but could be considered an interesting
%future work.

\section{Conclusion}
% Join is nice, but work remains.

We set out to investigate the usefulness of the join calculus as a
formalism for expressing programs that execute in a distributed environment with
limited resources and unreliable communication.

While we have realised that it isn't possible to constructively assert such an
abstract notion as \emph{usefulness} based on the small body of code that we have
produced, we have at least been able to report that we haven't met any
significant obstacles that obviously limit the elegance of the language.

Combining this with our reflections on the implementation impeding aspects of
the language, most notably the ones that involve global awareness, we believe
that a modified version without these problems would stand a decent chance at
becoming a programming language suitable for expressing programs in the
target domain of distributed embedded devices.
