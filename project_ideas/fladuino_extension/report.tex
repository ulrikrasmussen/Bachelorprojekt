\documentclass[a4paper, 11pt, oneside, report]{memoir}
\chapterstyle{culver}
\checkandfixthelayout

% Palatino font
\usepackage{palatino}

% Font and input encoding
\usepackage[T1]{fontenc}
\usepackage[utf8x]{inputenc}

% Babel (language)
\usepackage[english]{babel}

% Support for blackboard bold symbols.
%\usepackage{bbm}

% Nice math font
%\usepackage{eulervm}

% AMS-Math packages
%\usepackage{amsmath}
%\usepackage{amssymb}
%\usepackage{amsthm}

% For including bitmap graphics
%\usepackage{graphicx}

% Remove chapters from figure counters
%\counterwithout{figure}{chapter}
% Add sections instead
%\counterwithin{figure}{section}

\title{Extending Fladuino with support for expressing networks of embedded
devices}
\author{Ulrik Rasmussen \& Philip Lykke Carlsen}

\begin{document}

\maketitle

\section*{Project Contents}

The proposed project is an adaption of a bachelor's thesis from
2009\cite{fladuino} in which a framework called Fladuino is created, for
programming the Arduino devices in a declarative functional style. The project
is an adaption of the Haskell-embedded domain-specific language \emph{Flask},
which is used for programming sensor networks.

%In Fladuino, programs are expressed as acyclic dataflow graphs, in which data
%flows from sensors and through nodes in a network. Fladuino programs are said
%to be \emph{reactive}; the programmer doesn't need to keep track of when a
%specific input variable changes, but merely describes the relationship between
%the output-variables (the reactions) and the input-values (what is observed).

As it is, Fladuino only covers the programming of individual devices. If
multiple Arduinos were to communicate with each other in a network (for
instance, via bluetooth), possibly with different programs, there would be no
support for sanity-checking that the programs actually transmit valid (i.e.
type-safe) messages to each other. It would also be up to the programmer to
manually establish connections to the other units, and to define a common
protocol.

If cheap, embedded systems like the Arduino are to be deployed in vast numbers,
it would be a great improvement if Fladuino was extended to support (... to be
continued)

\bibliographystyle{plain}
\bibliography{bibliography}
\end{document}
