\documentclass[a4paper,draft]{article}
\usepackage[utf8]{inputenc}
\title{Join-calculus on embedded, distributed devices.}
\author{Ulrik Rasmussen \& Philip Carlsen}
\usepackage[usenames]{color}
\usepackage[T1]{fontenc}
\usepackage[danish]{babel}
\usepackage{lmodern}
\usepackage[pdftex]{graphics}
\usepackage{amsfonts}
\usepackage{amsmath}
\usepackage{amsthm}
\usepackage{amssymb}
\usepackage{latexsym}
\usepackage{verbatim}

\usepackage{fixme}

\begin{document} \maketitle \section*{Project Contents} The
intention of this project is to explore the possibilities
of applying the \emph{join process calculus} on the
programming of embedded -- possibly mobile -- devices, that
communicate in a network. 

We have chosen the Arduino platform as our target platform,
as there is wide range of boards available at low cost, and
because it is an open source project.

\section*{Project Result}
The project should produce a compiler that is capable of
compiling programs written in the core of the join calculus
language into C programs, which can then subsequently be
compiled by a suitable compiler to the arduino platform.


\section*{Project Motivation}
By expressing programs using a process calculus rather than
conventional thread programming, the programmer is freed
from having to specify how communication between processes
(and communications between different devices) takes place,
as that aspect of the program is abstracted away from the programmer and into the programming language. 

\section*{What's new} 
This has never been done before to our knowledge, as the existing implementations of the join calculus are extensions of O'Caml, which is way too big to fit into an arduino (with a whopping 64k working memory for the big boards)

Also, we think that the thing that comes closest is sensor
networks, as these are the usual manifestation of embedded
devices that communicate in a decentralised way. However,
the capabilities present in the devices are often limited
to the collection and relaying of data, and the data only
flows one way: to the machine that collects the data.

\section*{Project Risks}
Constructing an implementation of a programming language
with such advanced features as the join calculus is quite
an undertaking, and it will be necessary to limit the
borders of the project sufficiently, so that it will fit a
bachelors thesis.

\end{document}

