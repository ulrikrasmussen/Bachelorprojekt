\documentclass[a4paper,draft]{article}
\usepackage[utf8]{inputenc}
\title{Forslag til Bachelorprojekt}
\author{Philip Lykke Carlsen}
\usepackage[usenames]{color}
\usepackage[T1]{fontenc}
\usepackage[danish]{babel}
\usepackage{lmodern}
\usepackage[pdftex]{graphics}
\usepackage{amsfonts}
\usepackage{amsmath}
\usepackage{amsthm}
\usepackage{amssymb}
\usepackage{latexsym}
\usepackage{verbatim}

\usepackage{fixme}

\begin{document}
\maketitle
\section{Klassificering af tekster efter holdning.}

\paragraph{Problemet.}  Det er interessant at kende en persons meninger og
holdninger til andres meninger, hvis man fx ønsker at agere på basis den
holdning, der er flertal for, eller hvis man ønsker at servicere oplysninger,
der er relevante for mennesker med bestemte meninger og synspunkter, når de
forsøger at følge en debat om et bestemt emne.

Normalt er man ellers overladt helt til sig selv mht. at afgøre, hvilke
tekster, der er relevante for en, hvis man prøver at danne sig en mening på et
bestemt område. Typisk er det et privilegium, der er forbeholdt for
journalister, at have et mere end overfladisk kendskab til andre holdninger end
ens egne, da man ikke, hverken i hverdagen eller i fritiden, har tid til at
undersøge, om der findes nyheder eller artikler, der har interesse for én.  

\paragraph{Metoden.} 
\begin{itemize} \item Definer en ontologi for holdninger, der gør det
muligt for en person at udtrykke sin holdning (enig/uenig/etc.) om
indholdet af en tekst, og mulighed for at udtrykke sin holdning til
forfatteren af en tekst, så det fx er muligt at tilkendegive sin enighed
med en bestemt person på et bestemt område.

(RDF er oplagt)

eksempler: P er en person, T er en tekst, X er et område/semantisk domæne

\subitem Jeg er enig med P omkring X 

\subitem Jeg er uenig i holdningen
der er udtrykt i T 

\subitem Jeg synes at P er (in)kompetent på område X

\subitem Jeg synes at afsnit Y i tekst T udtrykker (positiv/negativ)
holdning til X

\item Undersøg, hvilke prædikater der skal til for at fremstille en model
til fornuftig klassifikation af tekster i forhold til holdninger.

\item Konstruer (eller specificer) et framework (webcrawler + holdningstilkendegivelsesfunktion), der kan aggregere
holdninger og holdninger til holdninger udtrykt på tværs af internettet

\item Konstruer en (Machine Learning) algoritme, der kan forklassificere
tekster på baggrund af forskellige parametre omkring en tekst, ved brug af
den aggregerede data fra forrige punkt. (fx forfatter, afsnits lighed i
ordvalg og formuleringer med andre tekster, offentliggørelsessted (url),
osv.)
\end{itemize}


\end{document}

