\% Synopsis in RST format

\section{Title}

Programming distributed embedded devices using join-calculus

\section{Problem statement}

Is it practical to use the join calculus to program mobile robots
that need to communicate in a dynamic setting, where there is no
guarantee of message delivery and not necessarily any centralised
communication available?

\section{Motivation}

Programmable robotic devices present many possibilities, for
instance in the field of self-reconfiguring modular robotics or in
swarm robotics. In fields like these, many autonomous robots has to
communicate and organise themselves to accomplish certain tasks.
There are no centralised means of communication available, and the
available communication channels may be unreliable. Also, programs
running on autonomous robots like these are inherently distributed
and asynchronous, which presents many challenges to the
programmer.

Many of the traditional tools available for programming such
devices are primitive, with concepts far from the problem domain
and closer to the actual hardware executing the program. Therefore,
it seems obvious that programming in a style that is more portable
and less explicit with regards to the actual communication
algorithms will benefit the programming of these devices. In both
these aspects, join calculus seems to be a promising abstraction of
parallel, communicating processes.

\section{Scope}

In this project we choose to limit our investigation to the matter
of expressing programs in join calculus, and avoid the matter of
actually executing the programs on the special hardware it was
meant for. Thus, this project will largely be about language design
and simulation.

\section{Tasks}

\begin{itemize}
\item
  Try to understand and document the relevant literature on
  join-calculus (2-3 weeks)

\item
  Design and document the programming language (\ensuremath{\sim}6-8
  weeks)

  \begin{quote}
  \begin{itemize}
  \item
    Decide on language constructs and semantics.
  \end{itemize}
  \end{quote}
\item
  Implement the programming language (2-3 weeks)

\item
  Construct sample programs (\ensuremath{\sim}2 weeks)

\item
  Implement a simulator that simulates robots in a 2D space
  (\ensuremath{\sim}2-3 weeks)

\item
  Document the performance of the sample programs and the efforts
  involved in constructing them. (1-2 weeks)

\item
  Compare join calculus to other process calculi (i.e. could any
  eventual challenges that we met have been avoided, had we used
  another process calculus?) (1 week)

\end{itemize}
\section{Product}

The product of the project will first of all be a report that
documents the effort, and secondly an implementation of our
programming language, probably in the shape of an EDSL with Haskell
as a meta language, along with a 2D simulation program, in which it
is possible to run programs on virtual robot units and observe the
communication and movements of the robots.

We will evaluate the usefulness of the calculus by constructing a
programming language based on the calculus, and a simulator that
captures the peculiarities of decentralised communication, along
with situations of unit failure and unit replacement.

\section{Literature}

The paper on join calculus.

(more to come)

