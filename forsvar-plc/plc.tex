% vim:set textwidth=80 fo+=tcroql:
\documentclass{beamer}
\mode<presentation>
{
  \usetheme{Copenhagen}
  \setbeamercovered{transparent}
}

\usepackage{pgfpages}
\setbeameroption{show notes on second screen}
\setbeameroption{always typeset second mode=true}

\usepackage[danish]{babel}
\usepackage[utf8]{inputenc}
\usepackage{palatino}
\usepackage[T1]{fontenc}
\title[Bachelorprojektforsvar]{Programming Mobile Embedded Devices using The Join Calculus} % (optional)
\author{Philip Lykke Carlsen \& Ulrik Rasmussen}
\institute{Datalogisk Institut, Københavns Universitet }
\date{24. Juni 2010}
\begin{document}

\frame{\titlepage}

\begin{frame}{Indhold}
  \tableofcontents
  % You might wish to add the option [pausesections]
\end{frame}

\begin{frame}{Problem.}
\begin{itemize}
\item<1->Motivation
\note<1->{

  Når man skal programmere mobile, indlejrede enheder såsom sensornetværk er man
  henvist til at bruge C eller tilsvarende og manuelt håndtere al
  enhedskommunikation.

  Der eksisterer proceskalkyler, de er bare enten synkrone eller ikke formelt
  definerede.

  }
\item<2->Join-calculus
\note<2->{

  Join-kalkylen er et forsøg på at skabe en asynkron proceskalkyle, der både er
  formelt defineret, asynkron og oprationaliserbar.

  Derfor har vi valgt at forsøge at konstruerer et programmeringssprog baseret
  på join-kalkylen, som vi har anvendt på et par eksempelprogrammer.

}
\end{itemize}
\end{frame}

\section{Programmeringssproget: Definition \& eksempler}
\subsection{Join Calculus}
\begin{frame}{Programmeringssproget.}
\end{frame}
\subsection{Use Cases}
\begin{frame}{Sensor Netværk.}

\end{frame}
\begin{frame}{Retransmission.}
\end{frame}
\section{Diskussion \& konklusion}
\begin{frame}{Diskussion.}
\end{frame}
\begin{frame}{Konklusion.}
\end{frame}

\begin{frame}{Motivation.}

  You can create overlays\dots
  \begin{itemize}
  \item using the \texttt{pause} command:
    \begin{itemize}
    \item
      First item.
      \pause
    \item
      Second item.
    \end{itemize}
  \item
    using overlay specifications:
    \begin{itemize}
    \item<3->
      First item.
    \item<4->
      Second item.
    \end{itemize}
  \item
    using the general \texttt{uncover} command:
    \begin{itemize}
      \uncover<5->{\item
        First item.}
      \uncover<6->{\item
        Second item.}
    \end{itemize}
  \end{itemize}
\end{frame}

\subsection{Second Subsection}

\begin{frame}{Make Titles Informative.}
\end{frame}

\begin{frame}{Make Titles Informative.}
\end{frame}



\begin{frame}{Summary}
  % Keep the summary *very short*.
  \begin{itemize}
  \item
    The \alert{first main message} of your talk in one or two lines.
  \item
    The \alert{second main message} of your talk in one or two lines.
  \item
    Perhaps a \alert{third message}, but not more than that.
  \end{itemize}

  % The following outlook is optional.
  \vskip0pt plus.5fill
  \begin{itemize}
  \item
    Outlook
    \begin{itemize}
    \item
      Something you haven't solved.
    \item
      Something else you haven't solved.
    \end{itemize}
  \end{itemize}
\end{frame}



\end{document}
