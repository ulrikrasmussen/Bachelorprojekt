% vim:set textwidth=80 fo+=tcroql:
\documentclass{beamer}
\mode<presentation>
{
  \usetheme[nat,dk]{Frederiksberg}
  %\setbeamercovered{transparent}
}

\usepackage{pgfpages}
\pgfpagesuselayout{4 on 1}[a4paper, landscape, border shrink=5mm]
%\setbeameroption{show notes on second screen}
\setbeameroption{show only notes}
%\setbeameroption{always typeset second mode=true}

\usepackage[danish]{babel}
\usepackage[utf8]{inputenc}
\usepackage{palatino}
\usepackage[T1]{fontenc}
\title[Bachelorprojektforsvar]{Programming Mobile Embedded Devices using The Join Calculus} % (optional)
\author{Philip Lykke Carlsen \& Ulrik Rasmussen}
\institute{Datalogisk Institut, Københavns Universitet}
\date{24. Juni 2010}
\begin{document}

\frame[plain]{\titlepage}

\begin{frame}{Indhold}
  \tableofcontents %[pausesections]
\end{frame}

\section{Problemet}
\begin{frame}{Problemet}
At definere et programmeringssprog, der med fordel kan anvendes på distribuerede
enheder.

\vspace{\stretch{1}}

\begin{itemize}
\item<2->Motivation: \\
Mangel på fornuftig, domænespecifik abstraktion.
\note<2>{

  Når man skal programmere mobile, indlejrede enheder, såsom sensornetværk er
  man henvist til at bruge C eller tilsvarende og manuelt håndtere al
  enhedskommunikation.

\vspace{1em}

  Der eksisterer processkalkyler, de er bare enten synkrone eller ikke formelt
  definerede.

\vspace{1em}

  Synkronisitet går ikke i en distribueret sammenhæng, da det er for kompliceret
  at implementere fornuftigt.

  }
\item<3->Join-calculus: \\
Asynkron, simpel, vist ækvivalent med $\pi$-calculus.
\note<3>{

  Join-kalkylen er et forsøg på at skabe en asynkron proceskalkyle, der både er
  formelt defineret, asynkron og oprationaliserbar.

  Derfor har vi valgt at forsøge at konstruerere et programmeringssprog baseret
  på join-kalkylen, som vi har anvendt på et par eksempelprogrammer.

  Udover grundkalkylen har vi en fortolker og simulator til et par udvidelser,
  beskrevet tidligere i artikler, der virkede relevante for problemdomænet.

}
\end{itemize}
\end{frame}

\section{Programmeringssproget}
\begin{frame}{Programmeringssproget}
Join-sproget er en udvidet udgave af grund-kalkylen.

\vspace{\stretch{1}}

\begin{itemize}
\item<2-> Forsøgt tilpasset til problemet.\\
Algebraiske datatyper, distribution, tid og usikker kommunikation.

\note<2>{

  Værdi: Nødvendig for et praktisk programmeringssprog (ellers skal alt
  simuleres gennem kernekonstruktionerne).

  \vspace{1em}

  Distribution: Relevant fordi den omhandler kommunikation mellem maskiner og
  giver mulighed for at afprøve kodemigration.

  \vspace{1em}

  Tid + usikker com.: Relevant da de gør det muligt deterministisk:\\
    \indent- at afgøre programadfæren ved udfald af kommunikation\\
    \indent- at definere timeouts

  }

\item<3-> Guidet og konkret mod simpel og abstrakt.\\
Opdelt i trin, der groft set svarer til hver udvidelse.

\note<3>{

  De oprindelige semantikker defineret for de enkelte udvidelser var relativt
  simple og enten ikke-deterministiske i deres eksekvering eller udvalgte
  reduktionsskridt fra kombinatorisk fremkomne mængder.

  \vspace{1em}

  Yderligere var særligt det at lave distribution og tid noget der blev dybt
  involveret, og måtte adskilles i afgrænsede ``pakker''.

  }
\end{itemize}

\end{frame}

\section{Evalueringen}

\begin{frame}{Use cases}
Vores vurdering af programmeringssproget er baseret på to eksempelprogrammer,
der omhandler:

\vspace{\stretch{1}}

\begin{itemize}
\item<2-> Sensornetværk,\\
Lagerhus med intelligente termometre.
\note<2>{

  Skitsér strukturen i netværket.\\
  Beret om query-systemet.

}
\item<3-> Garanteret levering, \\
Et protokolbibliotek med retransmission af beskeder.
\note<3>{

  Sammenlign med tcp.\\
  elegant brug med proxy-mønstret.

}
\end{itemize}
\end{frame}

\section{Resultat}

\begin{frame}{Resultat}
\begin{itemize}
\item<2-> Join-sproget skal viderudvikles før det kan bruges.
\note<2>{ Der er forskellige uafklarede forhold : migration, synkron tid.  }
\item<3-> Det virker ikke uegnet.
\note<3>{ Vi kan ikke konstruktivt entydigt konkludere egnethed, men vi har ikke
haft nogen nævneværdige problemer.

Tidsudvidelsen virker fornuftig. Samme med Algebraiske datatyper.

Dog savner man måske et lidt stærkere argument for at have kodemigration som en
del af sproget end at det er gavnligt for et applet-mønster.
}
\end{itemize}
\end{frame}

\end{document}
